\documentclass[sigconf]{acmart}

\setcopyright{none}
\acmConference[CSE 5095]{AI for Software Engineering}{Fall 2025}{University of Connecticut}

\begin{document}

\title{Supplementary Materials: Agentic Test Suite Refinement}
\author{Rye Howard-Stone}
\email{rye.howard-stone@uconn.edu}
\affiliation{\institution{University of Connecticut}}

\maketitle

\section{Appendix A: Prompt Variants}

The six prompt variants tested in Phase 2, ranging from minimal to maximal complexity:

\subsection{V1: Baseline}
\begin{verbatim}
Improve the test coverage for this repository.
Run the existing tests, identify gaps, and add
new tests to increase coverage.
\end{verbatim}

\subsection{V2: Baseline + Context}
\begin{verbatim}
You are an expert software testing engineer.
Analyze this Python repository and improve its
test coverage. First understand the codebase
structure, then systematically add tests for
uncovered code paths.
\end{verbatim}

\subsection{V3: Context + Coverage Feedback}
\begin{verbatim}
You are an expert software testing engineer.
Improve test coverage for this repository.

WORKFLOW:
1. Run `coverage run -m pytest` to measure current coverage
2. Run `coverage report --show-missing` to identify gaps
3. Write tests targeting uncovered lines
4. Re-run coverage to verify improvement
5. Repeat until coverage >= 80%
\end{verbatim}

\subsection{V4: Context + Mutation Feedback}
\begin{verbatim}
You are an expert software testing engineer.
Improve test coverage using mutation testing feedback.

WORKFLOW:
1. Run `mutmut run` to generate mutants
2. Run `mutmut results` to see surviving mutants
3. Write tests that kill surviving mutants
4. Focus on mutants in critical code paths
5. Verify mutant kill rate improves
\end{verbatim}

\subsection{V5: Context + Coverage + Mutations}
\begin{verbatim}
You are an expert software testing engineer.
Use both coverage analysis and mutation testing.

PHASE 1 - Coverage:
1. Run coverage analysis
2. Identify uncovered lines
3. Write tests for major gaps

PHASE 2 - Mutation Testing:
4. Run mutmut on covered code
5. Analyze surviving mutants
6. Strengthen assertions to kill mutants

PHASE 3 - Verification:
7. Re-run all tests
8. Verify coverage >= 80%
9. Verify mutation score improved
\end{verbatim}

\subsection{V6: Minimal}
\begin{verbatim}
Add tests to improve coverage.
\end{verbatim}

\section{Appendix B: Complete Metrics}

\subsection{Phase 1: Base vs Refine (12 configurations)}

\begin{table}[h]
\centering
\caption{Phase 1 Complete Results}
\begin{tabular}{llllrrr}
\hline
\textbf{Repo} & \textbf{Model} & \textbf{Strategy} & \textbf{Cov\%} & \textbf{Pass\%} & \textbf{Tests} & \textbf{Tokens} \\
\hline
schedule & Sonnet & base & 88 & 100.0 & 76 & 2.9M \\
schedule & Sonnet & refine & 85 & 72.5 & 216 & 4.9M \\
schedule & Opus & base & 91 & 100.0 & 94 & 3.2M \\
schedule & Opus & refine & 90 & 96.8 & 212 & 6.0M \\
mistune & Sonnet & base & 79 & 94.6 & 194 & 6.0M \\
mistune & Sonnet & refine & 72 & 85.0 & 258 & 5.6M \\
mistune & Opus & base & 76 & 94.3 & 376 & 4.2M \\
mistune & Opus & refine & 71 & 97.7 & 300 & 7.9M \\
click & Sonnet & base & 64 & 100.0 & 234 & 2.4M \\
click & Sonnet & refine & 64 & 91.4 & 334 & 8.2M \\
click & Opus & base & 67 & 91.9 & 212 & 10.6M$^*$ \\
click & Opus & refine & 66 & 95.5 & 298 & 3.3M \\
\hline
\end{tabular}
\label{tab:phase1-complete}

\small{$^*$Anomaly: Extended exploration cycles integrating with 190 existing baseline tests.}
\end{table}

\subsection{Phase 2: Six Prompt Variants (54 configurations)}

\begin{table}[h]
\centering
\caption{Phase 2 Experiment 1: Reduced Coverage Repositories}
\begin{tabular}{llrrrr}
\hline
\textbf{Repo} & \textbf{Variant} & \textbf{Cov\%} & \textbf{Pass} & \textbf{Fail} & \textbf{Tokens (M)} \\
\hline
click & V1 & 79 & 565 & 0 & 11.9 \\
click & V2 & 66 & 340 & 2 & 5.1 \\
click & V3 & 70 & 362 & 0 & 5.3 \\
click & V4 & 0 & 0 & 0 & 5.5 \\
click & V5 & 65 & 327 & 2 & 5.2 \\
click & V6 & 63 & 341 & 0 & 5.9 \\
\hline
mistune & V1 & 85 & 55 & 5 & 2.7 \\
mistune & V2 & 64 & 28 & 0 & 4.3 \\
mistune & V3 & 53 & 21 & 1 & 3.6 \\
mistune & V4 & 46 & 18 & 2 & 4.4 \\
mistune & V5 & 79 & 49 & 1 & 4.9 \\
mistune & V6 & 82 & 52 & 0 & 4.9 \\
\hline
schedule & V1 & 97 & 45 & 0 & 1.4 \\
schedule & V2 & 94 & 38 & 0 & 2.5 \\
schedule & V3 & 95 & 40 & 0 & 2.7 \\
schedule & V4 & 56 & 22 & 3 & 4.0 \\
schedule & V5 & 97 & 47 & 0 & 4.2 \\
schedule & V6 & 88 & 35 & 0 & 0.9 \\
\hline
\end{tabular}
\end{table}

\begin{table}[h]
\centering
\caption{Phase 2 Experiment 2: Low Existing Coverage Repositories}
\begin{tabular}{llrrrr}
\hline
\textbf{Repo} & \textbf{Variant} & \textbf{Cov\%} & \textbf{Pass} & \textbf{Fail} & \textbf{Tokens (M)} \\
\hline
python-box & V1 & 93 & 142 & 0 & 3.1 \\
python-box & V2 & 94 & 89 & 0 & 3.1 \\
python-box & V3 & 94 & 95 & 0 & 2.9 \\
python-box & V4 & 1 & 2 & 0 & 3.6 \\
python-box & V5 & 93 & 137 & 0 & 3.3 \\
python-box & V6 & 95 & 148 & 0 & 3.7 \\
\hline
colorama & V1 & 77 & 58 & 0 & 5.4 \\
colorama & V2 & 80 & 42 & 0 & 5.0 \\
colorama & V3 & 77 & 55 & 0 & 2.3 \\
colorama & V4 & 67 & 38 & 0 & 3.4 \\
colorama & V5 & 73 & 67 & 0 & 5.8 \\
colorama & V6 & 72 & 71 & 0 & 7.3 \\
\hline
boltons & V1 & 67 & 1205 & 12 & 6.4 \\
boltons & V2 & 68 & 1198 & 8 & 3.4 \\
boltons & V3 & 78 & 1245 & 5 & 7.7 \\
boltons & V4 & 65 & 1180 & 15 & 4.0 \\
boltons & V5 & 76 & 1232 & 3 & 7.6 \\
boltons & V6 & 70 & 1210 & 7 & 7.4 \\
\hline
\end{tabular}
\end{table}

\begin{table}[h]
\centering
\caption{Phase 2 Experiment 3: Bioinformatics Repositories}
\begin{tabular}{llrrrr}
\hline
\textbf{Repo} & \textbf{Variant} & \textbf{Cov\%} & \textbf{Pass} & \textbf{Fail} & \textbf{Tokens (M)} \\
\hline
dnaapler & V1 & 35 & 12 & 8 & 5.5 \\
dnaapler & V2 & 53 & 18 & 5 & 3.4 \\
dnaapler & V3 & 34 & 11 & 9 & 7.7 \\
dnaapler & V4 & 0 & 0 & 0 & 4.0 \\
dnaapler & V5 & 38 & 14 & 6 & 5.5 \\
dnaapler & V6 & 41 & 15 & 5 & 3.4 \\
\hline
fastqe & V1 & 95 & 28 & 2 & 2.9 \\
fastqe & V2 & 87 & 25 & 5 & 4.3 \\
fastqe & V3 & 37 & 8 & 12 & 3.5 \\
fastqe & V4 & 0 & 0 & 0 & 4.4 \\
fastqe & V5 & 0 & 0 & 0 & 4.9 \\
fastqe & V6 & 0 & 0 & 0 & 4.9 \\
\hline
pyfaidx & V1 & 50 & 45 & 15 & 5.2 \\
pyfaidx & V2 & 60 & 52 & 8 & 4.1 \\
pyfaidx & V3 & 55 & 48 & 12 & 6.2 \\
pyfaidx & V4 & 0 & 0 & 0 & 3.8 \\
pyfaidx & V5 & 45 & 40 & 18 & 5.8 \\
pyfaidx & V6 & 52 & 46 & 14 & 4.5 \\
\hline
\end{tabular}
\end{table}

\section{Appendix C: Statistical Analysis Code}

\begin{verbatim}
import numpy as np
from scipy.stats import binomtest, wilcoxon, mannwhitneyu

# V4 underperformed V1 in 9/9 repos
v4_worse = 9
result = binomtest(v4_worse, 9, p=0.5, alternative='greater')
print(f"Sign test (V4 < V1): p = {result.pvalue:.4f}")
# Output: p = 0.0020

# Variance decomposition
total_var = 151.7
between_repo_var = 132.0
between_variant_var = 5.2
print(f"Ratio: {between_repo_var/between_variant_var:.1f}x")
# Output: 25.5x

# Token efficiency V5 vs V6
v5_tokens = [5.15, 4.89, 4.25, 3.27, 5.83, 7.62]
v6_tokens = [5.86, 4.92, 0.94, 3.72, 7.32, 7.38]
v6_wins = sum(v6 < v5 for v6, v5 in zip(v6_tokens, v5_tokens))
result = binomtest(v6_wins, 6, p=0.5, alternative='greater')
print(f"Sign test: p = {result.pvalue:.4f}")
# Output: p = 0.8906 (NOT significant)

# Domain effects
general = [79, 85, 97, 93, 77, 67]
bio = [35, 95, 50]
u_stat, u_p = mannwhitneyu(bio, general, alternative='less')
print(f"Mann-Whitney U: p = {u_p:.4f}")
# Output: p = 0.1905 (NOT significant)
\end{verbatim}

\section{Appendix D: testScout Evaluation Results}

\begin{table}[h]
\centering
\caption{testScout Element Discovery Performance}
\begin{tabular}{llrrr}
\hline
\textbf{Website} & \textbf{Type} & \textbf{Elements} & \textbf{Time (s)} & \textbf{Success} \\
\hline
Wikipedia & Content Site & 280 & $<$5 & \checkmark \\
Hacker News & News Aggregator & 226 & $<$5 & \checkmark \\
GitHub Explore & Modern SPA & 220 & $<$5 & \checkmark \\
TodoMVC React & React App & 12 & $<$5 & \checkmark \\
\hline
\textbf{Total} & & \textbf{738} & & \textbf{100\%} \\
\hline
\end{tabular}
\end{table}

\begin{table}[h]
\centering
\caption{testScout Cost Comparison}
\begin{tabular}{lcc}
\hline
\textbf{Framework} & \textbf{Cost per Test} & \textbf{Notes} \\
\hline
testScout (Gemini) & \$0.01--\$0.05 & Pay-per-use \\
testScout (OpenAI) & \$0.10--\$0.50 & Pay-per-use \\
Selenium/Playwright & \$0 & Manual scripting \\
ZeroStep & \$49--\$249/mo & SaaS subscription \\
Checksum & \$99+/mo & SaaS subscription \\
\hline
\end{tabular}
\end{table}

\section{Appendix E: Anomaly Analysis}

\subsection{click/Opus/base: 10.6M Token Anomaly}

The click repository with Opus model using base strategy consumed 10.6M tokens---4.4$\times$ more than click/Sonnet/base (2.4M tokens). Investigation revealed:

\begin{enumerate}
\item \textbf{Existing test suite}: click had 190 baseline tests (vs. ~50 for schedule/mistune)
\item \textbf{Integration complexity}: Each new test required verifying compatibility with existing tests
\item \textbf{Extended exploration}: Opus model performed more thorough codebase analysis
\item \textbf{Iterative refinement}: Multiple rounds of test generation and debugging
\end{enumerate}

This suggests that repositories with large existing test suites may require fundamentally different strategies than greenfield testing.

\subsection{V4 Catastrophic Failures}

In 5/9 repositories, V4 (+Mutations) achieved $\leq$10\% coverage:

\begin{itemize}
\item \textbf{click}: 0\% (mutation testing never completed)
\item \textbf{python-box}: 1\% (mutmut errors on complex types)
\item \textbf{dnaapler}: 0\% (bioinformatics domain confusion)
\item \textbf{fastqe}: 0\% (emoji handling in mutants)
\item \textbf{pyfaidx}: 0\% (file format parsing mutants)
\end{itemize}

Common failure pattern: The agent enters mutation analysis loops, generating hundreds of mutants, then exhausts context attempting to address each one before writing basic coverage tests.

\section{Appendix F: JobsCoach Bug Reports}

Bugs detected by testScout during JobsCoach development:

\begin{enumerate}
\item \textbf{P0 - Route Ordering}: \texttt{/v2/jobs/fresh} caught by \texttt{/v2/jobs/\{job\_id\}} dynamic route
\item \textbf{P1 - Database Init}: \texttt{create\_all()} without \texttt{checkfirst=True} causing table exists errors
\item \textbf{P1 - Test Isolation}: 101 tests failing in suite but passing individually
\item \textbf{P1 - API Parameters}: \texttt{query} and \texttt{filters} silently ignored
\item \textbf{P2 - Duplicate Text}: ``Senior Senior Software Engineer'' in headlines
\item \textbf{P2 - Filter State}: Frontend/backend property mismatch
\end{enumerate}

All bugs were identified through systematic E2E testing and subsequently fixed.

\section{Appendix G: Reproducibility}

All experimental data, prompts, and analysis scripts are available at:

\begin{itemize}
\item \textbf{aTSR}: \url{https://github.com/rhowardstone/aTSR}
  \begin{itemize}
  \item \texttt{results/phase1/summary.json} - Phase 1 metrics
  \item \texttt{results/phase2/*/metrics.json} - Phase 2 metrics
  \end{itemize}
\item \textbf{testScout}: \url{https://github.com/rhowardstone/testscout}
\end{itemize}

\textbf{Model versions used}:
\begin{itemize}
\item Claude Sonnet 4.5 (claude-sonnet-4-5-20250901)
\item Claude Opus 4.1 (claude-opus-4-1-20250901)
\item Gemini 2.0 Flash (testScout experiments)
\end{itemize}

\end{document}
